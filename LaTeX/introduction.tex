\section*{Introduction}

Modularity is a pervasive concept in computer science, extending from the design of systems (Parnas 1972), to the design of software (Szyperski 1996). Modularity offers several advantages to both a developer and a user. In particular, functionality can be dynamically loaded and unloaded depending on the particular use case. Open source modular software precipitates the possibility of extensions contributed by a wide array of programmers, which can allow the software to morph into areas that weren’t anticipated early in development. In the statistics realm, \R (\R Core Team 2014) is a prime example of the virtues of modular programming. As of this writing, The Comprehensive \R Archive Network (CRAN) contains over 9000 source packages which can be installed and dynamically loaded in a particular
session as needed.

Modern web technologies have enabled a new generation of software packages that reside solely on the web, which eliminates the issue of local installation and helps abstract away some of the more challenging programming aspects of working directly with \R. Upon the release of RStudio’s Shiny (RStudio and Inc. 2014) it became easier for an \R-based analysis to be converted to an interactive web application. Several recent software packages have built upon Shiny to provide a web-based system based on R. One such package is iNZight Lite (Wild 2015) which attempts to expose students to data analysis without requiring programming knowledge. Like most web-based systems, this does not include reproducible \R code, which limits its usefulness in a scientific or academic setting. Another package is called Radiant (Nijs 2016), which is a web-based application with the aim of furthering business education and financial analysis. While the application is modular and extensible, it does require installation and hosting and is inundated with more features than necessary for an introductory student.  Partial fulfillment of requirements is noted in the table, as well as a measure of the complexity of functionality offered by default. For example, \R does have an associated Graphical User Interface (GUI), however this interface is very limited, thus only partially fulfilling the behavior of a GUI.

Unlike other web-based application, \FIRSTkit is done with the purpose to be used as a teaching component.

As a statistician in a School of Public Health, I saw first hand students struggling with \R, either by the difficulty or just they weren't interested. The goal of \FIRSTkit is to allow first time {\tt R} users, or users who have low interest in learning coding in perform basic statistics.

As a web-based application, this tool is immediately more familiar to students than a desktop application. The need for dealing with software licenses, installation configuration, and supported platforms has been eliminated. This allows students to spend more time working with the data and learning statistics than having to struggle to get the software running.

Unlike paid softwares, \FIRSTkit requires no software licenses or manual installation. \FIRSTkit 'functionality is focused on introductory statistics students, and nothing more (no extraneous functionality that must be navigated around to get to the content they need). \FIRSTkit provides students an opportunity to see the underlying functionality of the buttons, text boxes, and other UI elements, in an attempt to foster an interest in coding.


